\documentclass [12pt]{article}
\usepackage{float}
\usepackage[english]{babel}
\usepackage[T1]{fontenc}
\usepackage[utf8]{inputenc}
\usepackage{amsmath}
\usepackage{graphicx}
\usepackage{lmodern}
\usepackage{epsfig}
\usepackage[a4paper, total={6in, 9in}]{geometry}
\usepackage{caption}
\usepackage{hyperref}


\begin{document}
\noindent
If we suppose that the function which governs the evolution of the size is $$S_{i,j}=g\left( Age,\% light, soil, .... \right)\times f(t) .$$ That's a function $g$ with tree related variables and a function which will tell us how it grows with respect the time, e.g., $$S_{i,j}=g\times t^2$$ or $$S_{i,j}=g\times t^5$$ or even $$S_{i,j}=g\times e^t.$$


The main idea with these rules is to simplify the work to be done.  I divided  the rules on three lists ( in fact, the last one is only to remember us how the reproduction will be and it isn't a true list of rules): growth rules, which describes how the parameters affects (positively) the growth rate; and  death rules, which describes how the lack of certain necessities of the tree affects on its life expectancy. 
\section*{Rules}
\subsection*{Growth rules}
\begin{itemize}
\item 1st Rule: Determine the function $f(t)$.
\item 2nd Rule: How depends the quality of the soil on $g$ growth rate. In this rule the soil can be only neutral or positive ( no negative influence).
\item 3rd Rule: How the tree modifies the quality of the soil surround it.
\item 4th Rule: How the light percentage (from 0 to 1)  modifies again the $g$ growth rate. $0$ will be no light and $1$ optimal sunlight quantity (0 light is not a negative thing  in this function. The bad effect of lack of sunlight is determined in the next list of rules).
\item 5th Rule: How the actual age of the tree modifies the growth  rate $g$.
\end{itemize}

\subsection*{Death Rules}
\begin{itemize}
\item 1st Rule:  Max age (life expectancy) of each tree species. (We have to determine how it affects to the tree, I mean, as Alejandro told me, it could be in form of a  probability of disappear in each iteration once the value is reached ) (\textbf{Jordi})
\item 2nd Rule: How a soil poor in nutrients  affects the  life expectancy of a tree.  (\textbf{Jordi})
\item 3rd Rule: How the lack of sunlight or excess of it ( if it's possible) modifies the tree life expectancy.
\end{itemize}

\subsection*{Reproduction rules}


\begin{itemize}
\item 1st Rule: Probability of sending spores in each iteration. (In fact, it could be set artificially by us to see its effect on the total population )
\item 2nd Rule:  Quantity of spores that each species would send.
\item 3rd Rule: Probability that an spore in an empty place will become a tree. 
\end{itemize}

\end{document}